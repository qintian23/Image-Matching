\section{性能评估}

我们知道字符串匹配有精准匹配、部分匹配、前缀匹配、后缀匹配。当然,我并没有了解过字符匹配的性能评价。这里我设计了几个匹配效果的算法。

首先,对于匹配算法的性能我们希望它是鲁棒的、准确的、时间复杂度和空间度好的、稳定的,同时我们应该要测试一下这些算法的灵敏度。因此,我们应该要事先准备一个相关的评价模板,以及测试案例。
简而言之,匹配效果的描述应该由这两项构成:标准(鲁棒、准确率、泛化能力),性能(时间复杂度、空间复杂度、可并行化设计、稳定、灵敏)。
\begin{itemize}
    \item 评价模板
    \item 测试案例
\end{itemize}

FME的目的是作为一个模板来捕获用于基于分析的高级信息;效率变量、代价和收益,对于特征描述子而言,可用简单的属于来衡量效率,比如将计算代价和所占的内存与所获得的效益(包括精度、判别性、鲁棒性和
不变性)进行比较。方法消耗了多少时间、空间和资源呢?效率度量包括以下内容:
\begin{itemize}
    \item 代价:计算量、内存、时间和资源
    \item 收益:获得的精度、鲁棒性、和不变属性;
    \item 效率:相对于成本的收益
\end{itemize}

\subsection{鲁棒性评估}

这里首先给出一个标准表格,
\begin{table}[!htbp]
    \caption{一般的鲁棒性标准以及它们的属性}
    \label{tab:Robust}
    \centering
    \footnotesize% fontsize
    \begin{tabular}{|c|l|}
        \hline
        \textbf{鲁棒性} & \multicolumn{1}{c|}{\textbf{属性}} \\ \hline
        光照           & 分均匀光照、亮度、对比度、虚光                  \\ \hline
        颜色           & 色彩空间精度、颜色通道、颜色位深度                \\ \hline
        不完全性         & 混乱、遮挡、离群值、邻近、噪声、运动模糊、抖动、颤动       \\ \hline
        分辨率、精度       & 定位精度或位置、形状和厚度畸变、焦平面或深度、像素深度分辨率   \\ \hline
        几何失真         & 缩放、旋转、几何扭曲、反射、径向畸变、极向畸变          \\ \hline
        判别性、唯一性      & 质量                               \\ \hline
    \end{tabular}
\end{table}

\subsection{精准匹配}

不考虑形变的精确匹配,及其误差度量:基于像素的误差计算;考虑形变的精确匹配,形变函数后验概率。(误差不累计,以错误匹配/特征点提取总数计算,正确匹配认为落在精确点的n邻域)
\begin{equation}
    Matching_{error} = \frac{Keypoint_{error}}{Keypoint_{all}} 
    \label{equ:Matching_{error}}
\end{equation}
\begin{equation}
    Keypoint_{error} \in (Keypoint_{right}  \pm \epsilon) \\ 
    deformation(Keypoint_{error}) \in (Keypoint_{right}  \pm \epsilon)
    \label{equ:keypoint_{error}}
\end{equation}